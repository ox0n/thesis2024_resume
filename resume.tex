\documentclass[9pt, twocolumn, a4paper]{jsarticle_kijou}
%
\usepackage{amsmath,amssymb}
\usepackage{bm}
\usepackage[dvipdfmx]{graphicx}
\usepackage{ascmac}
\usepackage[deluxe]{otf} 
\usepackage[margin=19mm]{geometry}

%%%% タイトルのフォント設定
\makeatletter
\renewcommand{\title}[1]{\gdef\@title{\bfseries\sffamily#1}}
\makeatother

\title{上半身ヒューマノイドロボットによる胴体を活用した投球}
\author{
4年8組11番 奥村 健人\\ % 名前
明治大学 理工学部 機械情報工学科 複雑ロボットシステム研究室\\ %所属
指導教員 新山 龍馬 % 指導教員
}
\date{}
\pagestyle{empty}
\begin{document}
\maketitle


\section{はじめに}
ここでは卒業研究の目的,研究の背景,従来の研究を調べた結果について参考文献1)\UTF{FF5E}4)を引用しながら簡潔に述べる.どのような問題をどのように解決しようとしているのか、それによって、社会のどのように貢献できるか、等を明確に述べる。


\section{使用したロボットと投球動作の実装}
\subsection{ロボットの仕様}

% 本文の中で図表の説明は省略しないように.また、図表の説明で最初のものは太字(例\textbf{図1})で書くこと。

\subsection{投球モーションの生成}


\section{ロボットによる投球実験}
\subsection{腕のみの場合と協調動作を行った場合の比較}


\subsection{結果についての考察}



% \begin{figure}[t]
%  \centering
%  \includegraphics[width=0.5\textwidth]{example.eps}
%  \caption{Side view of fatigue life test rig}
%  \label{figure:1}
% \end{figure}

% \begin{table}[t]
%   \caption{Numbers of row and $K_A$ values}
%   \label{table:1}
%   \centering
%   \begin{tabular}{|c|c|c|c|c|}
%     \hline
%     No. of rows $J$  & 3  & 4 & 5 & 6  \\
%     \hline
%     $K_A$  & 1.0000   & 1.0000 & 1.1043 & 1.3292 \\
%     \hline
%   \end{tabular}
% \end{table}

\section{まとめ}



\section*{\small 参考文献}{
\small
\begin{enumerate}
\renewcommand{\labelenumi}{\arabic{enumi}).}
  \item 著者名、題名、巻-号、掲載雑誌(年)
  \item 著者名、本の題名、引用箇所、出版社(年)
  \item D
  \item 
\end{enumerate}
}

\end{document}